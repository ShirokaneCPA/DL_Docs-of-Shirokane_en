
\documentclass[dvipdfmx, a4paper,10pt,titlepage]{article}

\usepackage{etex}

\usepackage{multicol}
\usepackage{alltt}
\usepackage{fancybox}
\usepackage{fancyhdr}
\usepackage[dvipdfmx]{color}
\usepackage{xcolor}
\usepackage{enumitem}
\pagestyle{fancy}
\usepackage{tabularx}
\usepackage[dvipdfmx]{graphicx}
\usepackage[dvipsnames]{pstricks}
\usepackage{amssymb}
\usepackage{amsfonts}
\usepackage{latexsym}
\usepackage{amsmath}
\usepackage[dvipdfmx]{hyperref}
\usepackage{setspace}
\usepackage{lastpage}
\usepackage[all]{background}
%\usepackage{showframe}
\usepackage{tikz}
\usepackage{wallpaper}
\usepackage{lastpage}
\usepackage{wrapfig}
\usepackage{fontawesome}

\topmargin=0mm
\headheight=23mm
\headsep=65pt
\footskip=50pt
\oddsidemargin=55mm
\evensidemargin=20mm
\textwidth=108mm
\textheight=197mm
%\setstretch{0.8}


\ULCornerWallPaper{1.0}{fullcolor002.png}  %左上ロゴ表示


\definecolor{lightgreen}{cmyk}{ 0.2, 0, 0.2, 0}
\definecolor{maingreen}{cmyk}{ 1, 0.6, 0.3, 0.2}
\definecolor{dominant}{cmyk}{1, 0.1, 0.7, 0.2}

 \renewcommand{\headrulewidth}{11pt}% header rule
  \renewcommand{\headrule}{\hbox to 120mm{%
    \color{dominant} \leaders\hrule height11pt\hfil}}

\renewcommand{\contentsname}{} %「目次」の消去

%======見出し設定

\makeatletter
\renewcommand{\part}{\@startsection
{part}{0}{0mm}{25mm}{15mm}{\itshape \bfseries \Huge}}
\makeatother

\makeatletter
\renewcommand{\section}{\@startsection
{section}{3}{-60mm}{20mm}{5mm}{\color{dominant} \bfseries \LARGE}}
\makeatother


\makeatletter
\renewcommand{\subsection}{\@startsection
{subsection}{3}{-60mm}{5mm}{3mm}{\color{dominant} \Large}}
\makeatother

%======丸囲み数字の利用\maru{数字}

\newcommand{\maru}[1]{\ooalign{
\hfil\resizebox{.8\width}{\height}{#1}\hfil
\crcr
\raise.1ex\hbox{\large$\bigcirc$}}}

%========箇条書きのグリーンドット========
\newlist{coloritemize}{itemize}{1}
\setlist[coloritemize]{label=\textcolor{itemizecolor}{\textbullet}}
\colorlet{itemizecolor}{dominant}% Default colour for \item in itemizecolor



\begin{document}


% 共通項目

\newcommand{\nyuryokua}{Kabushiki Kaisha (KK) in Japan}
\newcommand{\nyuryokub}{Companies and KKs' organizations}
\newcommand{\nyuryokuc}{February 14, 2025}

% HPレジュメ用

\newcommand{\nyuryokud}{KKs' organizations}

% ここからAenda用


\newcommand{\nyuryokue}{ABC Ltd.}
\newcommand{\nyuryokuee}{{\small CEO} Mr. Aaaa Bbbb}
\newcommand{\nyuryokuf}{{\small Certified Public Accountant} SATOSHI SHITO}
\newcommand{\nyuryokug}{SHIP事業からの剰余金流用}
%\newcommand{\nyuryokuh}{September 12, 2016} %mtg時間を記載する場合はこれを使用
\newcommand{\nyuryokuh}{\today}

%=======================タイトルページ(表紙)==================

\title{{\color{dominant} \Huge \nyuryokua} \\ {\large \nyuryokub}}
\date{\nyuryokuc}
\author{\color{dominant}\sffamily \bfseries Shirokane CPA Firm}

\begin{titlepage}

\title{{\color{dominant} \Huge \nyuryokua} \\ {\large \nyuryokub}}
\date{\nyuryokuc}
\author{\color{dominant}\sffamily \bfseries Shirokane CPA Firm}

\maketitle  %HPテンプレート用にタイトルページの出力

\end{titlepage}

%=======================ヘッダー部================

\lhead{}  %ヘッダー左側を非表示にするため。%で機能停止すれば、章立てが表示される。
\rhead{\color{dominant} \nyuryokuc \ \ \ P.\ \thepage / \pageref{LastPage}}

\cfoot{}  %ページ底中央を非表示するため。%で機能停止すれが、ページ数が表示される。
\rfoot{\color{dominant} \sffamily \bfseries \href{http://www.shirokanecpa.com/}{http://www.shirokanecpa.com/} \\ \copyright 2025 Shirokane CPA Firm}

%=======================表題部==================

\vspace*{-20mm}
\begin{center}
\part*{\nyuryokud}
\end{center}
\vspace{15mm}

%\begin{table}[h]
%  \begin{tabular}{lcl}
%    To & : & \nyuryokue \\
%        &   & \nyuryokuee \\
%    From & : & \nyuryokuf \\
%    Subject & : & \nyuryokug \\
%    Date & : & \nyuryokuh \\
%  \end{tabular}
%\end{table}

 \renewcommand{\headrulewidth}{11pt}% secondgreen Line
 \hbox to 120mm{%
 \color{dominant} \leaders\hrule height11pt\hfil}

%=====================目次=======================

\vspace{18mm}
{\color{dominant} \LARGE \bfseries Contenets}
\vspace{-5mm}
\setcounter{tocdepth}{2}
\tableofcontents
\clearpage

%======================本文部分(以下を作成)======
\vspace{-20pt}


\section{Who can be an incorporator}

\begin{coloritemize}
\item Foreign nationals Can Also Be incorporators
\item Foreign Companies Can Also Be incorporators
\end{coloritemize}

An incorporator is an individual or entity responsible for planning the establishment of a company and carrying out the necessary incorporation procedures. Many documents required for company formation must include the incorporator's seal or, alternatively a signature.

Under this method of incorporation, the incorporator becomes the owner of the company once it is established. In the case of a KK, the incorporator assumes the role of a shareholder.

If you intend to establish a subsidiary, the parent company must act as the incorporator. Alternatively, an individual incorporator can first establish the company and then transfer all shares (equity) to the intended parent company.

Regardless of the method chosen, the incorporator plays a crucial role in the incorporation process. Therefore, I request that you provide accurate information about the incorporator.

\subsection{Foreign nationals Can Also Be incorporators}
A foreign national residing in Japan can serve as a incorporator and obtain a Japanese seal certificate, as they are legally recognized as a resident. This seal certificate allows them to complete the company establishment procedures.

If a foreign national living overseas (or a Japanese citizen residing abroad) wishes to become an incorporator, a signature certificate is required since a Japanese seal certificate cannot be issued. 

However, as long as a valid signature certificate is provided, the individual can act as a incorporator and establish a company, regardless of nationality.

\subsection{Foreign Companies Can Also Be incorporators}
Both individuals and companies can act as incorporators. A foreign company, defined as a company incorporated and registered outside Japan under the laws of its respective country, can also serve as a incorporator in Japan.

If a foreign company already has a commercial registration in Japan, it can proceed with incorporation procedures in the same manner as a Japanese company.

Even if a foreign company does not have a commercial registration in Japan, it can still act as an incorporator and establish a company. In this case, a certificate of registration from its country of incorporation and a signature certificate from its legal representative are required.

\section{Who Can be Representative Directors and Directors}

\begin{coloritemize}
\item What should be a company organization
\item Foreign nationals can be Representative Directors\/Directors
\item Affects opening of bank accounts
\end{coloritemize}

A GK must have at least one managing member to be formed. A legal entity can serve as a managing member, but if it does, it must designate an individual to carry out the associated duties.
At least one member, whether an individual or a legal entity, must be a representative member.

A KK must have at least one director and one of directors msut be a representative director.
Additionally, a KK can have a board of directors with at least two directors, a board of corporate auditors with at least two corporate auditors, and an accounting auditor.

\subsection{What should be a company organization}

During the start-up period, it is advisable to have a simple structure with managing members\/directors, including at least one representative member\/director, to minimize costs and enable quick decision-making for efficient management.

\subsection{Foreign nationals can be Representative Directors\/Directors}

Non-residents can serve as both a GK's managing partner and a KK's director. They do not need to be Japanese nationals.  

However, if a non-resident or non-Japanese national wishes to become an operating partner of a GK or a director of a KK, the incorporation process requires more complex documentation compared to that for a Japanese resident. (For details, see Company Formation Service.)

\subsection{Affects opening of bank accounts}

There is no issue if the representative or director of the company being established is a foreign national. However, if they are a non-resident without a Japanese address, they may face difficulties when applying to open a bank account.

If all of the company's directors are non-residents without a Japanese address, an account administrator with a Japanese address will be required.

If you need a Japanese resident to serve as a director of a KK or a managing member of a GK, you can appoint a nominee director. (For details, see Nominee Director Service.)

\clearpage

%======事務所&代表者紹介======

\clearpage  %ページ途中から続ける場合(Afterwordsの中)はキャンセル

\vspace*{-15mm}   %ページ途中から続ける場合はキャンセル
%\vspace{10mm}   %ページ途中から続ける場合にセット
\hfill
\begin{minipage}[t]{35mm}
%\begin{wrapfigure}[7]{r}{35mm}
%\includegraphics[width=35mm]{profilepic.png}
\includegraphics[width=35mm]{shirokanecpa.png}
%\end{wrapfigure}
\end{minipage}

\vspace{-40mm}    %横書き用
\section{My Profile}

\subsection{Personal Information}

\begin{table}[h]
%\begin{spacing}{0.9}
  \begin{tabular}{lll}
Name & {\sffamily \bfseries SATOHSI SHITO} & \\
Professional & \multicolumn{2}{l}{Certified Public Accountant, \textit{1999 (No. 020888)}} \\
\ licenses       & \multicolumn{2}{l}{Certified Tax Accountant, \textit{2013 (No. 127096)}} \\
Affiliation  & \multicolumn{2}{l}{Tokyo Tax Accountant Agency,} \\
 & \multicolumn{2}{l}{Tax Support committee at Shinagawa} \\
         & \multicolumn{2}{l}{Japan CPA Tax committee} \\
         & \multicolumn{2}{l}{Certified Support Agencies for Business Innovation} \\
\includegraphics[width=4mm]{phone.png} & {\sffamily \bfseries +81 80-1257-5877} & \\
\includegraphics[width=4mm]{mail.png} & {\sffamily \bfseries satoshi.shito@shirokanecpafirm.com} & \\
\includegraphics[width=4mm]{website.png} & \multicolumn{2}{l}{\href{https://www.shirokanecpa.com/}{En) https://www.shirokanecpa.com/}} \\
%            & \multicolumn{2}{l}{\href{https://www.shirokanecpafirm.com/}{Jp) https://www.shirokanecpafirm.com/}} \\
\includegraphics[width=4mm]{linkedin.png} & \multicolumn{2}{l}{https://www.linkedin.com/in/satoshishito/} \\    
\includegraphics[width=4mm]{facebook.png} & \multicolumn{2}{l}{Jp)https://www.facebook.com/shirokanecpafirmjp/} \\
            & \multicolumn{2}{l}{En) https://www.facebook.com/Shirokanecpafirm/} \\   
  \end{tabular}
%\end{spacing}
\end{table}

\subsection{Mission}
My objective is to offer adept guidance and steadfast support to entrepreneurs as well as small and medium-sized enterprises. Additionally, I strive to facilitate seamless communication between foreign nationals and the Japanese business community. Rooted in this mission, my commitment lies in nurturing growth and fostering success for all stakeholders as we collectively navigate the path forward into the future.

\subsection{Professional Career Summary}

\subsubsection{M\&A Excellence}
I founded Shirokane CPA Firm on June 25, 2013, in collaboration with esteemed professionals, establishing a boutique consultancy offering specialized financial and tax advisory services. My proficiency spans extensive financial and tax due diligence for Tokyo Stock Exchange-listed entities engaged in M\&A transactions. Moreover, I possess a proven track record in overseeing both statutory and non-statutory audits for discerning clientele.

\subsubsection{Cross-Border Collaboration \& Hong Kong IPO with IFRS}
A significant facet of my journey has centered on nurturing emerging IT venture enterprises, guiding them through intricate tax compliance and strategic business alignment. Leading the audit engagement for Microsoft Japan KK stands as a testament to my leadership. With precision, I directed a dedicated team while concurrently assuming responsibilities as the conduit for Japanese statutory audit, and as a component auditor within Deloitte Seattle's cross-border Microsoft Corp. project. This involvement facilitated seamless collaboration across international boundaries, interfacing with teams in Seattle, Shanghai, and Singapore, fortifying my global perspective on intricate cross-border financial frameworks.

Pioneering noteworthy accomplishments, I orchestrated the groundbreaking Japanese public offering on the Hong Kong Stock Exchange in 2011. Expertly overseeing a Japanese professional team, I meticulously synchronized efforts with the Hong Kong Deloitte team, culminating in the triumphant IPO of SBI Holding, Inc., a prominent force in the Japanese brokerage landscape listed on the Tokyo Stock Exchange.

\subsubsection{Transformation and Strategy in IPO Preparation}
Within Deloitte's Integrated Service Department, I masterminded the metamorphosis of PGM Golf Group in preparation for their Initial Public Offering (IPO). Harnessing my adeptness, I orchestrated strategic transitions and effectively steered the divestiture of credit holdings from Japanese financial institutions. My purview extended to executing post-IPO group audits for a diverse pool of up to 40 companies, and providing strategic counsel on Japan Internal Control Audit intricacies.

My illustrious 19-year CPA career spans partnerships with over 130 diverse companies, evaluating the intricacies of approximately 30 M\&A opportunities. These multifaceted experiences have endowed me with a profound comprehension of varied management paradigms, refining my discernment in critical decision-making, and accentuating the imperative of context-sensitive strategies. From shaping organizational frameworks to orchestrating seamless departmental dynamics and sculpting pragmatic crisis mitigation strategies, I am resolutely prepared to seamlessly integrate my comprehensive expertise and business acumen to surmount future challenges.


%\subsection{Personal Activities}

%{\bfseries \itshape Down-hill Skiing (off-piste):} \\
%From the top of following mountains, \\
%Mt. Fuji (\textit{the highest in Japan, 2012, 2013, and 2014}) \\
%Okuhodaka (\textit{3rd highest in Japan, 2013}) \\
%Harinoki (\textit{2,821m, 2008, 2009}), Tanigawa (\textit{1,977m, 2012-}) \\

%{\bfseries \itshape Traveling:} \\
%California and Oregon, U.S.  (\textit{1991})\\
%Austria (\textit{1998}), Australia (\textit{1998}) \\
%Poland and Estonia (\textit{2015}) \\
%Croatia, Bosnia and Herzegovina, and Montenegro (\textit{2016}) \\
%Domestic area \\
             
%{\bfseries \itshape Scuba Diving:} \\       
%Certification of Advanced Open Water Diver. Experience of \\
%TAKA Dive's 5days cruise  (\textit{1999}) and two 4days diving trips with catamaran yacht at Okinawa, Japan (\textit{2008, 2000}). \\






\end{document}

%=======テンプレ======

\section{}
\subsection{}

\begin{coloritemize}
\begin{spacing}{0.95}
  \item 
  \item 
  \item 
  \item 
%  \item 
\end{spacing}
\end{coloritemize}

%======右側折り返しの表(タイトル4文字のケース)28文字を調整して利用======
\begin{table}[h]
  \begin{tabular}{lcp{28em}}
    氏名 & : & Giovvani Pellone, ジォバニ・ペローネ \\
    住所 & : & 東京都目黒区上目黒3-12-23-308 \\
    生年月日 & : & 1964年7月18日 \\
    国籍 & : & イタリア \\
    VISA & : & 就労制限なし \\
    結婚 & : & 既婚(日本人) \\
    職業 & : & 工業デザイナー、クリエイティブ・ディレクター、グラフィック・デザイナー \\
  \end{tabular}
\end{table}

%======表
\begin{minipage}{120mm}
  \begin{tabular}{llll}
\hline
 事業の種類   & 主な収入 & 会計納税単位 & 備考 \\
\hline
  Airbnb事業 & ホストからの手数料収入 & 一般財団法人 & 地域活性化という公共性強調 \\
  ピザ屋 & 飲食店売上 & 株式会社\footnote{当初はアメリカン・ダイナーと同一法人で事業を開始し、後にアメリカン・ダイナーの会社を設立し、必要資産を営業譲渡する方法も考えられます。} & 初期投資が必要、従業員あり \\
  アメリカン・ダイナー & 飲食店売上 & 株式会社 & 初期投資が必要、従業員あり \\
  ヨガ・スタジオ事業 & レッスン料収入 & 個人事業主 & 河辺家の節税目的 \\
  河辺さん個人 & 一般財団法人からの給与所得 & 個人事業主 & \\
  & 飲食会社からの役員報酬 &  & \\
  \hline
  \end{tabular}
\end{minipage}